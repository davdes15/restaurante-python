%
% API Documentation for API Documentation
% Module main.BDCA
%
% Generated by epydoc 3.0.1
% [Wed Feb 13 08:04:34 2019]
%

%%%%%%%%%%%%%%%%%%%%%%%%%%%%%%%%%%%%%%%%%%%%%%%%%%%%%%%%%%%%%%%%%%%%%%%%%%%
%%                          Module Description                           %%
%%%%%%%%%%%%%%%%%%%%%%%%%%%%%%%%%%%%%%%%%%%%%%%%%%%%%%%%%%%%%%%%%%%%%%%%%%%

    \index{main \textit{(package)}!main.BDCA \textit{(module)}|(}
\section{Module main.BDCA}

    \label{main:BDCA}
Created on 19/12/2018

\textbf{Author:} a16daviddss




%%%%%%%%%%%%%%%%%%%%%%%%%%%%%%%%%%%%%%%%%%%%%%%%%%%%%%%%%%%%%%%%%%%%%%%%%%%
%%                               Functions                               %%
%%%%%%%%%%%%%%%%%%%%%%%%%%%%%%%%%%%%%%%%%%%%%%%%%%%%%%%%%%%%%%%%%%%%%%%%%%%

  \subsection{Functions}

    \label{main:BDCA:cargarCombo}
    \index{main \textit{(package)}!main.BDCA \textit{(module)}!main.BDCA.cargarCombo \textit{(function)}}

    \vspace{0.5ex}

\hspace{.8\funcindent}\begin{boxedminipage}{\funcwidth}

    \raggedright \textbf{cargarCombo}(\textit{list})

    \vspace{-1.5ex}

    \rule{\textwidth}{0.5\fboxrule}
\setlength{\parskip}{2ex}
    Anade todas las provincias al combo de provincias

\setlength{\parskip}{1ex}
    \end{boxedminipage}

    \label{main:BDCA:cargarmun}
    \index{main \textit{(package)}!main.BDCA \textit{(module)}!main.BDCA.cargarmun \textit{(function)}}

    \vspace{0.5ex}

\hspace{.8\funcindent}\begin{boxedminipage}{\funcwidth}

    \raggedright \textbf{cargarmun}(\textit{listmun}, \textit{prov})

    \vspace{-1.5ex}

    \rule{\textwidth}{0.5\fboxrule}
\setlength{\parskip}{2ex}
    Anade todos los municipios de una determinada provincia al combobox de 
    municipios

\setlength{\parskip}{1ex}
    \end{boxedminipage}

    \label{main:BDCA:recuperarprovincia}
    \index{main \textit{(package)}!main.BDCA \textit{(module)}!main.BDCA.recuperarprovincia \textit{(function)}}

    \vspace{0.5ex}

\hspace{.8\funcindent}\begin{boxedminipage}{\funcwidth}

    \raggedright \textbf{recuperarprovincia}(\textit{provincia})

    \vspace{-1.5ex}

    \rule{\textwidth}{0.5\fboxrule}
\setlength{\parskip}{2ex}
    Recoge el id de una provincia de la base de datos y lo devuelve 
    restandole 1 para que pueda ser usado en el combobox

\setlength{\parskip}{1ex}
    \end{boxedminipage}

    \label{main:BDCA:recuperarmunicipio}
    \index{main \textit{(package)}!main.BDCA \textit{(module)}!main.BDCA.recuperarmunicipio \textit{(function)}}

    \vspace{0.5ex}

\hspace{.8\funcindent}\begin{boxedminipage}{\funcwidth}

    \raggedright \textbf{recuperarmunicipio}(\textit{provincia}, \textit{localidad})

    \vspace{-1.5ex}

    \rule{\textwidth}{0.5\fboxrule}
\setlength{\parskip}{2ex}
    recibiendo un id de provincia y un municipo obtiene el id del municipio
    y los id maximo y minimo de municipios de esa provincia, con esos datos
    traduce el id del municipio a la posicion que este tendra en el 
    combobox y devuelve esta posicion

\setlength{\parskip}{1ex}
    \end{boxedminipage}


%%%%%%%%%%%%%%%%%%%%%%%%%%%%%%%%%%%%%%%%%%%%%%%%%%%%%%%%%%%%%%%%%%%%%%%%%%%
%%                               Variables                               %%
%%%%%%%%%%%%%%%%%%%%%%%%%%%%%%%%%%%%%%%%%%%%%%%%%%%%%%%%%%%%%%%%%%%%%%%%%%%

  \subsection{Variables}

    \vspace{-1cm}
\hspace{\varindent}\begin{longtable}{|p{\varnamewidth}|p{\vardescrwidth}|l}
\cline{1-2}
\cline{1-2} \centering \textbf{Name} & \centering \textbf{Description}& \\
\cline{1-2}
\endhead\cline{1-2}\multicolumn{3}{r}{\small\textit{continued on next page}}\\\endfoot\cline{1-2}
\endlastfoot\raggedright b\-d\- & \raggedright \textbf{Value:} 
{\tt \texttt{'}\texttt{Provincias}\texttt{'}}&\\
\cline{1-2}
\raggedright c\-o\-n\-e\-x\- & \raggedright \textbf{Value:} 
{\tt {\textless}sqlite3.Connection object{\textgreater}}&\\
\cline{1-2}
\raggedright c\-u\-r\- & \raggedright \textbf{Value:} 
{\tt {\textless}sqlite3.Cursor object{\textgreater}}&\\
\cline{1-2}
\raggedright \_\-\_\-p\-a\-c\-k\-a\-g\-e\-\_\-\_\- & \raggedright \textbf{Value:} 
{\tt \texttt{'}\texttt{main}\texttt{'}}&\\
\cline{1-2}
\end{longtable}

    \index{main \textit{(package)}!main.BDCA \textit{(module)}|)}
