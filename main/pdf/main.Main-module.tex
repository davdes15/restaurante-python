%
% API Documentation for API Documentation
% Module main.Main
%
% Generated by epydoc 3.0.1
% [Wed Feb 13 08:04:26 2019]
%

%%%%%%%%%%%%%%%%%%%%%%%%%%%%%%%%%%%%%%%%%%%%%%%%%%%%%%%%%%%%%%%%%%%%%%%%%%%
%%                          Module Description                           %%
%%%%%%%%%%%%%%%%%%%%%%%%%%%%%%%%%%%%%%%%%%%%%%%%%%%%%%%%%%%%%%%%%%%%%%%%%%%

    \index{main \textit{(package)}!main.Main \textit{(module)}|(}
\section{Module main.Main}

    \label{main:Main}
Created on 19/12/2018

\textbf{Author:} a16daviddss




%%%%%%%%%%%%%%%%%%%%%%%%%%%%%%%%%%%%%%%%%%%%%%%%%%%%%%%%%%%%%%%%%%%%%%%%%%%
%%                               Variables                               %%
%%%%%%%%%%%%%%%%%%%%%%%%%%%%%%%%%%%%%%%%%%%%%%%%%%%%%%%%%%%%%%%%%%%%%%%%%%%

  \subsection{Variables}

    \vspace{-1cm}
\hspace{\varindent}\begin{longtable}{|p{\varnamewidth}|p{\vardescrwidth}|l}
\cline{1-2}
\cline{1-2} \centering \textbf{Name} & \centering \textbf{Description}& \\
\cline{1-2}
\endhead\cline{1-2}\multicolumn{3}{r}{\small\textit{continued on next page}}\\\endfoot\cline{1-2}
\endlastfoot\raggedright W\-H\-E\-R\-E\-\_\-A\-M\-\_\-I\- & \raggedright \textbf{Value:} 
{\tt abspath(dirname(\_\_file\_\_))}&\\
\cline{1-2}
\end{longtable}


%%%%%%%%%%%%%%%%%%%%%%%%%%%%%%%%%%%%%%%%%%%%%%%%%%%%%%%%%%%%%%%%%%%%%%%%%%%
%%                           Class Description                           %%
%%%%%%%%%%%%%%%%%%%%%%%%%%%%%%%%%%%%%%%%%%%%%%%%%%%%%%%%%%%%%%%%%%%%%%%%%%%

    \index{main \textit{(package)}!main.Main \textit{(module)}!main.Main.Restaurante \textit{(class)}|(}
\subsection{Class Restaurante}

    \label{main:Main:Restaurante}

%%%%%%%%%%%%%%%%%%%%%%%%%%%%%%%%%%%%%%%%%%%%%%%%%%%%%%%%%%%%%%%%%%%%%%%%%%%
%%                                Methods                                %%
%%%%%%%%%%%%%%%%%%%%%%%%%%%%%%%%%%%%%%%%%%%%%%%%%%%%%%%%%%%%%%%%%%%%%%%%%%%

  \subsubsection{Methods}

    \label{main:Main:Restaurante:__init__}
    \index{main \textit{(package)}!main.Main \textit{(module)}!main.Main.Restaurante \textit{(class)}!main.Main.Restaurante.\_\_init\_\_ \textit{(method)}}

    \vspace{0.5ex}

\hspace{.8\funcindent}\begin{boxedminipage}{\funcwidth}

    \raggedright \textbf{\_\_init\_\_}(\textit{self})

\setlength{\parskip}{2ex}
\setlength{\parskip}{1ex}
    \end{boxedminipage}

    \label{main:Main:Restaurante:salir}
    \index{main \textit{(package)}!main.Main \textit{(module)}!main.Main.Restaurante \textit{(class)}!main.Main.Restaurante.salir \textit{(method)}}

    \vspace{0.5ex}

\hspace{.8\funcindent}\begin{boxedminipage}{\funcwidth}

    \raggedright \textbf{salir}(\textit{self}, \textit{widget}, \textit{data}={\tt None})

\setlength{\parskip}{2ex}
\setlength{\parskip}{1ex}
    \end{boxedminipage}

    \label{main:Main:Restaurante:about}
    \index{main \textit{(package)}!main.Main \textit{(module)}!main.Main.Restaurante \textit{(class)}!main.Main.Restaurante.about \textit{(method)}}

    \vspace{0.5ex}

\hspace{.8\funcindent}\begin{boxedminipage}{\funcwidth}

    \raggedright \textbf{about}(\textit{self}, \textit{widget})

\setlength{\parskip}{2ex}
\setlength{\parskip}{1ex}
    \end{boxedminipage}

    \label{main:Main:Restaurante:hide}
    \index{main \textit{(package)}!main.Main \textit{(module)}!main.Main.Restaurante \textit{(class)}!main.Main.Restaurante.hide \textit{(method)}}

    \vspace{0.5ex}

\hspace{.8\funcindent}\begin{boxedminipage}{\funcwidth}

    \raggedright \textbf{hide}(\textit{self}, \textit{widget})

\setlength{\parskip}{2ex}
\setlength{\parskip}{1ex}
    \end{boxedminipage}

    \label{main:Main:Restaurante:hidedialog}
    \index{main \textit{(package)}!main.Main \textit{(module)}!main.Main.Restaurante \textit{(class)}!main.Main.Restaurante.hidedialog \textit{(method)}}

    \vspace{0.5ex}

\hspace{.8\funcindent}\begin{boxedminipage}{\funcwidth}

    \raggedright \textbf{hidedialog}(\textit{self}, \textit{widget}, \textit{data}={\tt None})

\setlength{\parskip}{2ex}
\setlength{\parskip}{1ex}
    \end{boxedminipage}

    \label{main:Main:Restaurante:evtlog}
    \index{main \textit{(package)}!main.Main \textit{(module)}!main.Main.Restaurante \textit{(class)}!main.Main.Restaurante.evtlog \textit{(method)}}

    \vspace{0.5ex}

\hspace{.8\funcindent}\begin{boxedminipage}{\funcwidth}

    \raggedright \textbf{evtlog}(\textit{self}, \textit{window}, \textit{event})

\setlength{\parskip}{2ex}
\setlength{\parskip}{1ex}
    \end{boxedminipage}

    \label{main:Main:Restaurante:login}
    \index{main \textit{(package)}!main.Main \textit{(module)}!main.Main.Restaurante \textit{(class)}!main.Main.Restaurante.login \textit{(method)}}

    \vspace{0.5ex}

\hspace{.8\funcindent}\begin{boxedminipage}{\funcwidth}

    \raggedright \textbf{login}(\textit{self}, \textit{widget}, \textit{data}={\tt None})

    \vspace{-1.5ex}

    \rule{\textwidth}{0.5\fboxrule}
\setlength{\parskip}{2ex}
    Recoge el usuario y la contraseña de los cuadros de texto y se los 
    envía al metodo login, si son correctos accede a la ventana principal 
    y guarda el id del camarero en una variable, si no sacará un aviso 
    dicienndo que el usuario o la contraseña son incorrectos

\setlength{\parskip}{1ex}
    \end{boxedminipage}

    \label{main:Main:Restaurante:registrar}
    \index{main \textit{(package)}!main.Main \textit{(module)}!main.Main.Restaurante \textit{(class)}!main.Main.Restaurante.registrar \textit{(method)}}

    \vspace{0.5ex}

\hspace{.8\funcindent}\begin{boxedminipage}{\funcwidth}

    \raggedright \textbf{registrar}(\textit{self}, \textit{widget})

    \vspace{-1.5ex}

    \rule{\textwidth}{0.5\fboxrule}
\setlength{\parskip}{2ex}
    Recoge el usuario y la contraseña dos veces, si no hay nada en blanco 
    y las contraseñas coinciden la envía al metodo registrar de BDres, si
    el usuario es registrado se limpian los campos, si se da un error en el
    registro o en una de las comprobaciones mencionadas anteriormente

\setlength{\parskip}{1ex}
    \end{boxedminipage}

    \label{main:Main:Restaurante:reserva}
    \index{main \textit{(package)}!main.Main \textit{(module)}!main.Main.Restaurante \textit{(class)}!main.Main.Restaurante.reserva \textit{(method)}}

    \vspace{0.5ex}

\hspace{.8\funcindent}\begin{boxedminipage}{\funcwidth}

    \raggedright \textbf{reserva}(\textit{self}, \textit{widget}, \textit{data}={\tt None})

\setlength{\parskip}{2ex}
\setlength{\parskip}{1ex}
    \end{boxedminipage}

    \label{main:Main:Restaurante:updatemun}
    \index{main \textit{(package)}!main.Main \textit{(module)}!main.Main.Restaurante \textit{(class)}!main.Main.Restaurante.updatemun \textit{(method)}}

    \vspace{0.5ex}

\hspace{.8\funcindent}\begin{boxedminipage}{\funcwidth}

    \raggedright \textbf{updatemun}(\textit{self}, \textit{widget}, \textit{data}={\tt None})

\setlength{\parskip}{2ex}
\setlength{\parskip}{1ex}
    \end{boxedminipage}

    \label{main:Main:Restaurante:altacli}
    \index{main \textit{(package)}!main.Main \textit{(module)}!main.Main.Restaurante \textit{(class)}!main.Main.Restaurante.altacli \textit{(method)}}

    \vspace{0.5ex}

\hspace{.8\funcindent}\begin{boxedminipage}{\funcwidth}

    \raggedright \textbf{altacli}(\textit{self}, \textit{widget}, \textit{data}={\tt None})

\setlength{\parskip}{2ex}
\setlength{\parskip}{1ex}
    \end{boxedminipage}

    \label{main:Main:Restaurante:cleanCli}
    \index{main \textit{(package)}!main.Main \textit{(module)}!main.Main.Restaurante \textit{(class)}!main.Main.Restaurante.cleanCli \textit{(method)}}

    \vspace{0.5ex}

\hspace{.8\funcindent}\begin{boxedminipage}{\funcwidth}

    \raggedright \textbf{cleanCli}(\textit{self})

\setlength{\parskip}{2ex}
\setlength{\parskip}{1ex}
    \end{boxedminipage}

    \label{main:Main:Restaurante:cleanProd}
    \index{main \textit{(package)}!main.Main \textit{(module)}!main.Main.Restaurante \textit{(class)}!main.Main.Restaurante.cleanProd \textit{(method)}}

    \vspace{0.5ex}

\hspace{.8\funcindent}\begin{boxedminipage}{\funcwidth}

    \raggedright \textbf{cleanProd}(\textit{self})

\setlength{\parskip}{2ex}
\setlength{\parskip}{1ex}
    \end{boxedminipage}

    \label{main:Main:Restaurante:altaprod}
    \index{main \textit{(package)}!main.Main \textit{(module)}!main.Main.Restaurante \textit{(class)}!main.Main.Restaurante.altaprod \textit{(method)}}

    \vspace{0.5ex}

\hspace{.8\funcindent}\begin{boxedminipage}{\funcwidth}

    \raggedright \textbf{altaprod}(\textit{self}, \textit{widget}, \textit{data}={\tt None})

\setlength{\parskip}{2ex}
\setlength{\parskip}{1ex}
    \end{boxedminipage}

    \label{main:Main:Restaurante:selectcli}
    \index{main \textit{(package)}!main.Main \textit{(module)}!main.Main.Restaurante \textit{(class)}!main.Main.Restaurante.selectcli \textit{(method)}}

    \vspace{0.5ex}

\hspace{.8\funcindent}\begin{boxedminipage}{\funcwidth}

    \raggedright \textbf{selectcli}(\textit{self}, \textit{widget})

    \vspace{-1.5ex}

    \rule{\textwidth}{0.5\fboxrule}
\setlength{\parskip}{2ex}
    Carga los datos del cliente seleccionado en los campos de texto y 
    combobox, para la carga de los combobox se hace uso de los metodos 
    recuperarprovincia y recuperarmunicipio del modulo BDCA

\setlength{\parskip}{1ex}
    \end{boxedminipage}

    \label{main:Main:Restaurante:selectprod}
    \index{main \textit{(package)}!main.Main \textit{(module)}!main.Main.Restaurante \textit{(class)}!main.Main.Restaurante.selectprod \textit{(method)}}

    \vspace{0.5ex}

\hspace{.8\funcindent}\begin{boxedminipage}{\funcwidth}

    \raggedright \textbf{selectprod}(\textit{self}, \textit{widget})

\setlength{\parskip}{2ex}
\setlength{\parskip}{1ex}
    \end{boxedminipage}

    \label{main:Main:Restaurante:cambiobtn}
    \index{main \textit{(package)}!main.Main \textit{(module)}!main.Main.Restaurante \textit{(class)}!main.Main.Restaurante.cambiobtn \textit{(method)}}

    \vspace{0.5ex}

\hspace{.8\funcindent}\begin{boxedminipage}{\funcwidth}

    \raggedright \textbf{cambiobtn}(\textit{self}, \textit{widget})

\setlength{\parskip}{2ex}
\setlength{\parskip}{1ex}
    \end{boxedminipage}

    \label{main:Main:Restaurante:ocupar}
    \index{main \textit{(package)}!main.Main \textit{(module)}!main.Main.Restaurante \textit{(class)}!main.Main.Restaurante.ocupar \textit{(method)}}

    \vspace{0.5ex}

\hspace{.8\funcindent}\begin{boxedminipage}{\funcwidth}

    \raggedright \textbf{ocupar}(\textit{self}, \textit{widget}, \textit{data}={\tt None})

\setlength{\parskip}{2ex}
\setlength{\parskip}{1ex}
    \end{boxedminipage}

    \label{main:Main:Restaurante:verfact}
    \index{main \textit{(package)}!main.Main \textit{(module)}!main.Main.Restaurante \textit{(class)}!main.Main.Restaurante.verfact \textit{(method)}}

    \vspace{0.5ex}

\hspace{.8\funcindent}\begin{boxedminipage}{\funcwidth}

    \raggedright \textbf{verfact}(\textit{self}, \textit{widget})

\setlength{\parskip}{2ex}
\setlength{\parskip}{1ex}
    \end{boxedminipage}

    \label{main:Main:Restaurante:verlineas}
    \index{main \textit{(package)}!main.Main \textit{(module)}!main.Main.Restaurante \textit{(class)}!main.Main.Restaurante.verlineas \textit{(method)}}

    \vspace{0.5ex}

\hspace{.8\funcindent}\begin{boxedminipage}{\funcwidth}

    \raggedright \textbf{verlineas}(\textit{self}, \textit{widget}, \textit{data}={\tt None})

\setlength{\parskip}{2ex}
\setlength{\parskip}{1ex}
    \end{boxedminipage}

    \label{main:Main:Restaurante:addlineaf}
    \index{main \textit{(package)}!main.Main \textit{(module)}!main.Main.Restaurante \textit{(class)}!main.Main.Restaurante.addlineaf \textit{(method)}}

    \vspace{0.5ex}

\hspace{.8\funcindent}\begin{boxedminipage}{\funcwidth}

    \raggedright \textbf{addlineaf}(\textit{self}, \textit{widget})

\setlength{\parskip}{2ex}
\setlength{\parskip}{1ex}
    \end{boxedminipage}

    \label{main:Main:Restaurante:pagar}
    \index{main \textit{(package)}!main.Main \textit{(module)}!main.Main.Restaurante \textit{(class)}!main.Main.Restaurante.pagar \textit{(method)}}

    \vspace{0.5ex}

\hspace{.8\funcindent}\begin{boxedminipage}{\funcwidth}

    \raggedright \textbf{pagar}(\textit{self}, \textit{widget})

\setlength{\parskip}{2ex}
\setlength{\parskip}{1ex}
    \end{boxedminipage}

    \label{main:Main:Restaurante:backup}
    \index{main \textit{(package)}!main.Main \textit{(module)}!main.Main.Restaurante \textit{(class)}!main.Main.Restaurante.backup \textit{(method)}}

    \vspace{0.5ex}

\hspace{.8\funcindent}\begin{boxedminipage}{\funcwidth}

    \raggedright \textbf{backup}(\textit{self}, \textit{widget})

    \vspace{-1.5ex}

    \rule{\textwidth}{0.5\fboxrule}
\setlength{\parskip}{2ex}
    El metodo crea una copia de seguridad de la base de datos y despues nos
    abre una el directorio en el que se encuentra

\setlength{\parskip}{1ex}
    \end{boxedminipage}

    \label{main:Main:Restaurante:set_style}
    \index{main \textit{(package)}!main.Main \textit{(module)}!main.Main.Restaurante \textit{(class)}!main.Main.Restaurante.set\_style \textit{(method)}}

    \vspace{0.5ex}

\hspace{.8\funcindent}\begin{boxedminipage}{\funcwidth}

    \raggedright \textbf{set\_style}(\textit{self})

    \vspace{-1.5ex}

    \rule{\textwidth}{0.5\fboxrule}
\setlength{\parskip}{2ex}
    Change Gtk+ Style

\setlength{\parskip}{1ex}
    \end{boxedminipage}

    \index{main \textit{(package)}!main.Main \textit{(module)}!main.Main.Restaurante \textit{(class)}|)}
    \index{main \textit{(package)}!main.Main \textit{(module)}|)}
